
\documentclass[runningheads,a4paper]{llncs}
%\documentclass{llncs}
%

\usepackage{llncsdoc}
\usepackage{amssymb}
\usepackage{amsmath}
\usepackage[utf8x]{inputenc}

\usepackage{url}
%\usepackage{moreverb}
\usepackage{verbatim}
\usepackage{enumerate}
\usepackage{listings}
%\usepackage{fancyvrb}
%\usepackage[table]{xcolor} % to use cellcolor
\usepackage{xspace}
\usepackage{graphicx}
\usepackage{caption}
\usepackage{multirow}
\usepackage{xcolor}


\renewenvironment{description}[1][0pt]
  {\list{}{\labelwidth=0pt \leftmargin=#1
   \let\makelabel\descriptionlabel}}
  {\endlist}

\newcommand{\tth}{\texttt{ttH\_dilep}\xspace}
\newcommand{\ttDilepKinFit}{\texttt{ttDilepKinFit}\xspace}
\newcommand{\ttLoop}{\texttt{Loop}\xspace}
\newcommand{\ttDoCuts}{\texttt{DoCuts}\xspace}
\newcommand{\dilep}{\texttt{dilep}\xspace}
\newcommand{\ttbar}{$t\bar{t}$\xspace}

\begin{document}

\title{Removing Inefficiencies From Scientific Code: A Case Of The ATLAS Experiment 
\thanks{Detalhes financiamento...}}

%\end{comment}


\titlerunning{Removing Inefficiencies From Scientific Code}

% the name(s) of the author(s) follow(s) next
%
% NB: Chinese authors should write their first names(s) in front of
% their surnames. This ensures that the names appear correctly in
% the running heads and the author index.
%
\author{André Pereira\inst{1,2} \and António Onofre \inst{1,2} \and \\ Alberto Proença\inst{1}}
%
\authorrunning{Pereira, Onofre, and Proença}
% (feature abused for this document to repeat the title also on left hand pages)

% the affiliations are given next; don't give your e-mail address
% unless you accept that it will be published
\institute{Universidade do Minho, Portugal \and LIP-Minho, Portugal 
\\ \{ampereira, aproenca\}@di.uminho.pt, antonio.onofre@cern.ch}


\maketitle              % typeset the title of the contribution
%\thispagestyle{empty}
%\pagestyle{empty}

\begin{abstract}

This paper presents a set of methods and techniques for removing inefficiencies in scientific code. A scientific application, of the ATLAS experiment, is used as a case study. Here, we implement and test a set of optimizations to deal with performance inefficiencies in a organized approach: identification, removal at application development and at application runtime. These optimizations consider code, algorithm and data structure design, parallelization in shared memory systems, and runtime configurations for NUMA environments. These optimizations target different multithreading and multiprocess combinations, and CPU affinities. We expect that this communication will aid scientists to become more aware of common efficiency pitfalls in scientific code.

\keywords{High Performance Computing, Efficiency Optimization}
\end{abstract}
%

\section{Introduction}
\label{introduction}

The European Organization for Nuclear Research (CERN) is a consortium of 20 European countries aiming to operate the largest High Energy Physics (HEP) experiments in the world. The instrumentation used in nuclear and particle physics research is essentially formed by particle accelerators and detectors. The Large Hadron Collider (LHC) particle accelerator speeds up groups of particles close to the speed of light, in opposite directions, inducing a controlled collision the detectors core. The detectors record various characteristics of the resultant particles of each collision (an event), such as energy and momentum, which originate from complex decay processes of the collided protons. The purpose of these experiments is to test the HEP theories, such as the Standard Model, by confirming or discovering new particles and physics.

The ATLAS experiment is one of the seven particle detectors at CERN whose main goals are to confirm the existence of the top quarks and the Higgs boson in the Standard model. Approximately 600 million collisions occur every second at the LHC. Particles produced in head-on proton collisions interact with the detectors, generating massive amounts of raw data. It is estimated that all the combined detectors produce 25 petabytes of data per year, and it is expected to grow after the ongoing LHC upgrade \cite{LIP:Ibergrid}. This data then passes a set of processing and refinement stages until it is ready to be used to reconstruct the events by specific data analysis code, which may vary according to the HEP theory being researched. Several research groups work in event reconstruction in the same experiment, enforcing positive competition to produce quality results in a fast and consistent way.

These factors enforce the need to process more data, more accurately, in less time. Research groups often opt to invest on larger computing clusters to improve the quality of their research results. However, most scientific code was not designed and/or developed for an efficient use of the available computational resources. If these applications were adequately tuned (or redesigned), the event analysis throughput could be massively increased. An efficient parallel application can significantly improve its performance at a much lower cost, as shown in \cite{Msc:AMP}.

This paper addresses inefficiencies in two stages of the data analysis application: the code development and application runtime. In the former, inefficiencies in the algorithm coding and data structuring are pinpointed and several solutions are suggested, based on a quantitative analysis of the bottlenecks. The latter identifies inefficiencies in threads accessing remote shared memory, and gives hints to overcome these limitations.

This paper is organized as follows: section \ref{problem_and_app} briefly presents the top quark and Higgs boson decay process and introduces a short characterization of the data analysis application used as case study; in section \ref{identification} the code inefficiencies are identified, analysed, and removed, with a final shared memory parallelization proposal; in section \ref{removal}, runtime inefficiencies of the parallelization are identified and possible alternatives suggested, concluding with an assessment of the core affinity impact; finally, section \ref{conclusion} concludes the paper and proposes some future work.

\section{Top Quark and Higgs Decay}
\label{problem_and_app}

In the LHC, two proton beams are accelerated close to the speed of light in opposite directions, set to collide inside a specific particle detector. From this head-on collision results a chain reaction of decaying particles, and most of the final particles react with the detector allowing their characteristics to be recorded. One of the experiments being conducted at the ATLAS detector is related to the studies of top quark and Higgs boson properties. Figure \ref{fig:ttbar} presents the schematic representation of the top quark decay (addressed as the \ttbar system).

\begin{figure}[!htp]
	\begin{center}
		\includegraphics[scale=0.4]{images/ttbar_higgs.png}
		\caption{Schematic representation of the \ttbar system and Higgs boson decay.}
		\label{fig:ttbar}
	\end{center}
\end{figure}

The ATLAS detector can record the characteristics of the bottom quarks, detected as a jet of particles, and leptons (muon and electron). However, neutrinos do not interact with the detector and, therefore, their characteristics are not recorded. Since the top quark reconstruction requires the neutrinos, their characteristics are analytically determined with the known information of the system, in a process known as kinematical reconstruction. However, the \ttbar system may not have a possible reconstruction. The reconstruction has a degree of certainty associated, which determines its quality.

The amount of Bottom quark jets and leptons detected may vary between events, due to other reactions occurring alongside the top quark decay. As represented in figure \ref{fig:ttbar}, 2 jets and 2 leptons are needed to reconstruct the \ttbar system, but the input data for an event may have many more of these particles associated. It is necessary to reconstruct the system for every combination of 2 jets and 2 leptons in the input data (referred only as combination). Then, only the most accurate reconstruction of each event is considered.

The Higgs boson is reconstructed from the two jets that it decays to, but only if the \ttbar system as a possible reconstruction. This adds at least two more jets to the event information, increasing the number of possible combinations, as they are the same as the \ttbar system jets. The Higgs boson reconstruction uses the jets that were not needed in the \ttbar system. The overall quality of the event reconstruction depends on the quality of both \ttbar system and Higgs boson reconstructions.

The ATLAS detector has an experimental resolution that induces an error of 2\% in each measure of the particle characteristics. This error is propagated into the \ttbar system and Higgs reconstructions, affecting their accuracy. To improve the quality of the reconstructions a random variation to the particles parameters is applied, with a maximum magnitude of 2\%, a given amount of times and chose the event reconstruction with the highest accuracy. The quality of the reconstructions and application execution time are directly proportional to the amount of variations performed per combination. The goal is to do as many variations as possibly within a reasonable time frame.

The \tth application was designed to reconstruct of the \ttbar system and Higgs boson, as explained above. The application flow is presented in figure \ref{fig:flow}. Each event on the input file is individually loaded to a single global state shared among the application and the LipMiniAnalysis library, and it is overwritten every time a new event is loaded. The event is then submitted to a series of cuts, which filters events that are not suited for reconstruction. When an event reaches the cut 20 the \ttbar system and Higgs boson are reconstructed in a function named \ttDilepKinFit. If the \ttbar system reconstruction fails, the current combination is discarded and the next is processed. If an event has a possible reconstruction it passes the final cut and its final information is stored.

\begin{figure}[!htp]
	\begin{center}
		\includegraphics[scale=0.4]{images/graf_abstract_flow_with_kinfit.png}
		\caption{Schematic representation for the \tth application flow.}
		\label{fig:flow}
	\end{center}
\end{figure}

\section{Coding Inefficiencies}
\label{identification}

Inefficiency removal is a two stage iterative process, where bottlenecks are identified and later removed. First, the application is profiled and analysed to identify the critical sections of the code that take longer to compute. Then, the critical section is optimized by modifying the code, algorithm, or parallelization. The identification of critical sections can be automated by using third party tools, such as gprof\cite{GPROF}, Callgrind\cite{Callgrind}, or VTune\cite{Intel:VTune}, which produce reports listing the percentage of time spent in each of the application functions. A more detailed analysis can be obtained using tools similar to PAPI\cite{PAPI}, where hardware counters are used to quantify cache miss rates, executed floating point instructions, and other low level information.

The test environment used in both this section and section \ref{removal} is a dual-socket system with two Intel Xeon E5-2670v2\cite{Intel:e5v2} with 10 cores, with hardware support for 20 simultaneous threads, at 2.5 GHz each, 256 KB L2 cache per core and 25 MB shared L3 cache, with 64 GB DDR3 RAM. The K-Best measurement heuristic was adopted to ensure that the only the best, but consistent, time measurements are considered. Software wise, the GNU Compiler version 4.8.2 with \textit{O3} optimizations enabled and ROOT 5.34/17 were used. A 5\% interval was used for a \textit{k} of 4, with a minimum of 12 and maximum of 24 time measurements.

Profiling the data analysis code using Callgrind, the \ttDilepKinFit was identified as the most time consuming function, taking 99\% of the execution time for 1024 variations. \tth execution with this amount of variations was considered reasonable for all efficiency measurements unless stated otherwise, without compromising the application execution time.

A preliminary computational analysis concluded that the application is compute bound on the testbed system, where accesses to the system RAM memory are not a limiting factor with a ratio of 7 instructions per fetched byte for 1024 variations.

An analysis of the code showed two major inefficiencies restricting the performance. The pseudo-random number generation is consuming a large part of the \ttDilepKinFit execution time due to coding inefficiencies. The supplied LipMiniAnalysis data structuring prevents processing in parallel events from the same input file, leading to inefficient parallel code. These two issues are further detailed in the next subsections.

\subsection{Pseudo-Random Number Generation Inefficiencies}

Pseudo-random number generators (PRNGs) are common in many Monte Carlo simulation and reconstruction applications. A good PRNG deterministically generates uniform numbers with a long period, its produced values pass a set of randomness tests and, in HPC, it must be efficient and scalable. Repeatability is ensured by providing a seed to the PRNG prior to number generation, due to their deterministic execution.

The variation for the kinematical and Higgs reconstructions is based on applying a random offset to the current particles characteristics. This offset has a maximum magnitude of $\pm1\%$ of the original value and is computed by a PRNG. An analysis of the callgraph produced for 256 variations (higher variations made the Callgrind execution time infeasible) showed that 63\% of \tth execution time was spent on the PRNG. However, 23\% of the time was spent defining a new seed for the PRNG. Figure \ref{fig:prng256} presents the callgraph for the \ttDilepKinFit function of \tth.

\begin{figure}[!htp]
	\begin{center}
		\includegraphics[scale=0.5]{images/prng_256_edited.png}
		\caption{Callgraph subset of the \ttDilepKinFit most time consuming functions for 256 variations per combination.}
		\label{fig:prng256}
	\end{center}
\end{figure}

An analysis of the code showed that the application uses a PRNG available in ROOT, which uses the Mersenne Twister algorithm\cite{MersenneTwister}, resetting the seed for every parameter variation. The Mersenne Twister period is approximately $4.3 * 10^{6001}$, while the maximum amount of pseudo random numbers generated by the application, for the input file used and 1024 variations, is $3 * 10^9$, making the seed reset unnecessary. The removal of this inefficiency granted a 71\% performance improvement.

\subsection{Data Structure Inefficiencies}
\label{data_inef}

Once removed the PRNG seed reset inefficiency, the \ttDilepKinFit still remained the critical region in the application, with no apparent code inefficiency. The most obvious solution is to process in parallel several events from the same input file. However, the function in LipMiniAnalysis that loads events from a file into memory assigns a single global space. This data structure contains information that is modified during the event reconstruction process, and it is overwritten for every event loaded. Changing the data structure to support multiple events in memory simultaneously, and loading all events in the input file at the beginning of the data analysis, would allow the parallel processing of events with low overhead. However, as mentioned in section \ref{problem_and_app} many data analysis applications depend on LipMiniAnalysis preventing any modifications to its structure, so alternative solutions were explored.

\subsection{Alternative Parallel Approaches}

Next step to improve the code execution time is to parallelize \ttDilepKinFit. Note that it is not possible to parallelize the whole event processing since only one is loaded at a time and part of its information is stored in LipMiniAnalysis toolbox. Besides not allowing this parallelization, reading events individually is more inefficient than reading all events at once, where in the former slower random reads are made on the hard drive and in the latter the fast sequential reads are used.

Parallelizing \ttDilepKinFit implies modifying its flow. Currently, for each different combination of jets and leptons from an event, the processed data of each variation of the detector measurements is overwritten. A new data structure is required to hold all combinations of each event. Picking a lepton/jet combination depends on all previous chosen combinations, which serializes the construction of the data structure. Each parallel task (indivisible work segment) selects a combination with variations still to compute, then varies the particles parameters, performs the kinematical reconstruction, and attempts to reconstruct the Higgs boson. A parallel merge is performed after all combinations are computed to get the most accurate reconstruction for the event. Figure \ref{fig:SeqPipeline} presents the sequential and parallel workflow for \ttDilepKinFit.

\begin{figure}[!htp]
	\begin{center}
		\raisebox{-0.5\height}{\includegraphics[scale=0.5]{images/sequential_kinfit.png}}
		\raisebox{-0.5\height}{\includegraphics[scale=0.5]{images/parallel_kinfit.png}}
		\caption{Schematic representation of the \ttDilepKinFit workflows: sequential (left) and parallel (right).}
		\label{fig:SeqPipeline}
	\end{center}
\end{figure}

A shared memory parallelization using OpenMP\cite{OpenMP} was devised, as it is the best approach for single shared memory systems. The parallel tasks are grouped into threads, which holds the best reconstruction to minimize the complexity of the merge by reducing through all the threads instead of tasks. The amount of tasks for each thread is balanced dynamically by the OpenMP scheduler, as the workload is irregular since the Higgs boson reconstruction execution is not always computed. Each thread has a private PRNG initialized with different seeds to avoid correlation between the numbers generated.

\begin{figure}[!htp]
	\begin{center}
		\includegraphics[scale=0.55]{charts/speedup_non_pointer_omp.png}
		\caption{Speedup for the \tth original parallel version of the application.}
		\label{fig:non_pointer_speedup}
	\end{center}
\end{figure}

Figure \ref{fig:non_pointer_speedup} presents the speedups for different number of parallel threads. The purpose of the 1 thread test is to evaluate the parallelization overhead. The best efficient implementation occurs when using 2 and 4 threads, where the application is using almost all resources at each used core. The best overall performance occurs for 40 threads, but it only offers a speedup of 8.8, underusing the available 20 physical cores. Note that there is no significant overhead due to NUMA\footnote{NUMA, Non-Unified Memory Access, since each Xeon device has its own memory controller with attached RAM: RAM access time for each core differs as the RAM is connected to the same device or the neighbour Xeon.} accesses, as seen by the constant increase in performance from 10 to 16 threads. For more than 20 threads all available resources on both CPU devices.

% ------------- pointer version abaixo -----------------

The lack of scalability beyond a low number of parallel threads suggests that inefficiencies may still affect the application, probably caused by the parallelization overhead. Intel's VTune was used to search for hotspots (bottlenecks) on the parallel \tth, since this tool is best suited for profiling parallel applications while providing a user friendly graphic interface. A preliminary analysis showed that the application was spending 20\% of the execution time building the combination data structure for 256 variations.

An analysis of the coded data structure showed that inefficiencies were affecting the performance in specific situations. Data that is read-only on the parallel section is being replicated in each element of the data structure. If the elements were to share a pointer to such data, the overhead of constructing the data structure would be reduced. However, this could lead to worse cache management, due to cache line invalidations, since the application is accessing data on memory more frequently, and the data structuring did not efficiently separate read-only data from read/write data. This is particularly critical in NUMA environments, where communication costs are higher. This was implemented and tested (addressed as \textit{pointer version}), with its speedups ploted in figure \ref{fig:pointer_speedup}. The reference value for the speedup computation is still the same sequential version.

\begin{figure}[!htp]
	\begin{center}
		\includegraphics[scale=0.55]{charts/speedup_pointer_omp.png}
		\caption{Speedup for \tth parallel pointer and non-pointer implementations.}
		\label{fig:pointer_speedup}
	\end{center}
\end{figure}

As expected, the best speedup occurs when using only one CPU device. The performance degradation from 8 to 10 threads (on the same device) may be explained by the increase of concurrent accesses to the shared L3 cache. However, this implementation is more efficient than the non-pointer implementation when using only one device. Note that the superlinear speedups is due to the reduction in the data that each thread has to process, making it suitable to be stored in the private L2 cache of each core, avoiding the slower accesses to the L3 cache.

\section{Identification and Removal of Runtime Inefficiencies}
\label{removal}

The identification and removal of inefficiencies follows the same iterative approach presented in section \ref{identification}.

\subsection{First Iteration}

Without the sensibility provided by the tests in section \ref{identification}, a scientist would incur in the pitfall of using all available cores (and even all hardware threads) on the system hoping that it would provide the best performance. While it may be true for the non-pointer implementation, it would inefficiently use the system computational resources, and using the single device highly efficient pointer implementation would provide a even greater waste.

Since the pointer implementation is the most efficient but only when using a single device, using multiple processes may provide better efficiency. As it was not possible to refactor LipMiniAnalysis to change the event information from a global to a private state, an MPI implementation was discarded due to the communication overhead necessary in each event processing.

A characteristic of particle reconstructions at CERN is that the processing is made by executing the application on a vast set of 1GB input files. The system resources can be efficiently used if it is performed a careful balancing of the input files across a set of application processes, producing the same goal of an MPI implementation but with no need for communicating between processes. A dispatcher was devised, which takes a set of input files and creates a given amount of \tth processes. It is then responsible for dispatching the files to the different processes in a queue-like approach, and monitors their execution. A set of 20 input files was considered for testing purposes, with different configurations of processes and threads per process.

\begin{figure}[!htp]
	\begin{center}
		\raisebox{-0.5\height}{\includegraphics[scale=0.7]{images/sequential_kinfit.png}}
		\raisebox{-0.5\height}{\includegraphics[scale=0.7]{images/parallel_kinfit.png}}
		\caption{Schematic representation of the \ttDilepKinFit sequential (left) and parallel (right) workflows.}
		\label{fig:Sched}
	\end{center}
\end{figure}

Figure \ref{fig:Sched} presents the speedups using 2, 4, 5, 8, and 10 processes for different thread configurations to fill one or both CPU devices.

\begin{itemize}
	\item At runtime
	\begin{itemize}
		\item Thread affinity experiments vs standard OpenMP affinity (force bad affinity examples that may occur to compare the performance?)
		\item Using all available cores is not always profitable
		\item Many processes/threads combinations, also testing thread affinity
	\end{itemize}
\end{itemize}

\section{Conclusion}
\label{conclusion}

This paper presents a study of the inefficiencies in scientific code, using a particle reconstruction analysis application as a case study. Top quark and Higgs boson studies require reconstructing from measurements of a very large number of particle collisions, performed weekly by the \tth application on terabytes of data. A faster and more accurate analysis of the data allows to better reconstruct more particle collisions and to improve the quality of the research results.

We identified and removed inefficiencies in different stages of the application: when designing the code, and during its submission for execution. In the code design, inefficiencies in the pseudo-random number generation were tackled, a common component of most simulation and analysis applications, which led to 71\% performance improvement. In data structure design, by analysing the factors limiting the particle reconstruction parallelization, and presenting and testing two alternative solutions for shared memory environments. The non-pointer implementation had a constant scaling on a dual-socket NUMA system, while the pointer implementation was much more efficient using a single CPU.

At application runtime, a multiprocess approach using the more efficient parallel implementation tackled its inefficiencies on NUMA systems, providing a maximum speedup of 69.3. An efficient control on the thread affinity of this implementation provided a performance improvement up to 90\%. However, the fluctuation in performance and the dependencies on many system characteristics prevented the definition f a generalized heuristic to aid to control the best affinity for the application, for any computing system.

\begin{figure}[!htp]
	\begin{center}
		\includegraphics[scale=0.55]{charts/events_per_sec.png}
		\caption{Throughput of events processed for the original sequential \tth,with no and 1024 variations (\textit{var}), and for the parallel pointer and multiprocess implementations (with \textit{t} being threads and \textit{p} processes).}
		\label{fig:events_sec}
	\end{center}
\end{figure}

Figure \ref{fig:events_sec} plots the number of events processed per second in the original sequential application without variations of the particle characteristics, for 1024 variations, and for the parallel pointer and multiprocess implementations for 1024 variations. Using a single process, the throughput improves by a factor of 12.8, from 5 to 64 events per second, using a single CPU device. The best overall performance, for the multiprocess implementation with 8 processes with 5 threads each, provided an improvement by a factor of 113 on event throughput, from 5 to 560 events per second.

There are other important aspects that improve code efficiency, which were not taken into consideration. The compiler, and respective optimization options, can have a big impact in application performance, where different options may be best suited for certain applications than others. Rules for writing vectorized code were not presented but may improve the efficiency when processing large amounts of data, as the compiler is not able to perform most code modifications necessary to take advantage of vector instructions. Using efficient numeric libraries\cite{MKL,BLAS} when required usually has a big impact on performance. There are guides that provide rules and present techniques for developing efficient code\cite{Intel:Optimization,Intel:DevOptimization,NUMA}.

We hope to improve the sensibility of scientists to the efficiency pitfalls common in scientific code, to help develop more efficient and performing applications. The performance of the \tth application was improved by a factor of 113, for the test system used, helping physicists to execute more particle collisions with a more refined reconstruction process, which efficiently uses the available computational resources.

Even with the interesting improvements to the application efficiency, some directions of future research were identified. The scheduler could be improved to automatically predict the best process/thread configuration for each system by analysing a set of micro-benchmarks or the application itself on a small input, and ultimately identify the best core affinity scheme. Also, the application efficiency could be improved using hardware accelerators, balancing the workload among accelerators and CPU devices in heterogeneous systems. Frameworks for parallelization and workload balancing for heterogeneous systems can be analysed and tested with this case study.


% Acknowledgements

\paragraph{Acknowledgments:}


\bibliographystyle{splncs}
\bibliography{my}


\end{document}
